\documentclass[]{article}
\usepackage{lmodern}
\usepackage{amssymb,amsmath}
\usepackage{ifxetex,ifluatex}
\usepackage{fixltx2e} % provides \textsubscript
\ifnum 0\ifxetex 1\fi\ifluatex 1\fi=0 % if pdftex
  \usepackage[T1]{fontenc}
  \usepackage[utf8]{inputenc}
\else % if luatex or xelatex
  \ifxetex
    \usepackage{mathspec}
  \else
    \usepackage{fontspec}
  \fi
  \defaultfontfeatures{Ligatures=TeX,Scale=MatchLowercase}
\fi
% use upquote if available, for straight quotes in verbatim environments
\IfFileExists{upquote.sty}{\usepackage{upquote}}{}
% use microtype if available
\IfFileExists{microtype.sty}{%
\usepackage{microtype}
\UseMicrotypeSet[protrusion]{basicmath} % disable protrusion for tt fonts
}{}
\usepackage[margin=1in]{geometry}
\usepackage{hyperref}
\hypersetup{unicode=true,
            pdftitle={Clustering Consumers via Topic Modeling: Application to Advertising},
            pdfborder={0 0 0},
            breaklinks=true}
\urlstyle{same}  % don't use monospace font for urls
\usepackage{color}
\usepackage{fancyvrb}
\newcommand{\VerbBar}{|}
\newcommand{\VERB}{\Verb[commandchars=\\\{\}]}
\DefineVerbatimEnvironment{Highlighting}{Verbatim}{commandchars=\\\{\}}
% Add ',fontsize=\small' for more characters per line
\usepackage{framed}
\definecolor{shadecolor}{RGB}{248,248,248}
\newenvironment{Shaded}{\begin{snugshade}}{\end{snugshade}}
\newcommand{\KeywordTok}[1]{\textcolor[rgb]{0.13,0.29,0.53}{\textbf{#1}}}
\newcommand{\DataTypeTok}[1]{\textcolor[rgb]{0.13,0.29,0.53}{#1}}
\newcommand{\DecValTok}[1]{\textcolor[rgb]{0.00,0.00,0.81}{#1}}
\newcommand{\BaseNTok}[1]{\textcolor[rgb]{0.00,0.00,0.81}{#1}}
\newcommand{\FloatTok}[1]{\textcolor[rgb]{0.00,0.00,0.81}{#1}}
\newcommand{\ConstantTok}[1]{\textcolor[rgb]{0.00,0.00,0.00}{#1}}
\newcommand{\CharTok}[1]{\textcolor[rgb]{0.31,0.60,0.02}{#1}}
\newcommand{\SpecialCharTok}[1]{\textcolor[rgb]{0.00,0.00,0.00}{#1}}
\newcommand{\StringTok}[1]{\textcolor[rgb]{0.31,0.60,0.02}{#1}}
\newcommand{\VerbatimStringTok}[1]{\textcolor[rgb]{0.31,0.60,0.02}{#1}}
\newcommand{\SpecialStringTok}[1]{\textcolor[rgb]{0.31,0.60,0.02}{#1}}
\newcommand{\ImportTok}[1]{#1}
\newcommand{\CommentTok}[1]{\textcolor[rgb]{0.56,0.35,0.01}{\textit{#1}}}
\newcommand{\DocumentationTok}[1]{\textcolor[rgb]{0.56,0.35,0.01}{\textbf{\textit{#1}}}}
\newcommand{\AnnotationTok}[1]{\textcolor[rgb]{0.56,0.35,0.01}{\textbf{\textit{#1}}}}
\newcommand{\CommentVarTok}[1]{\textcolor[rgb]{0.56,0.35,0.01}{\textbf{\textit{#1}}}}
\newcommand{\OtherTok}[1]{\textcolor[rgb]{0.56,0.35,0.01}{#1}}
\newcommand{\FunctionTok}[1]{\textcolor[rgb]{0.00,0.00,0.00}{#1}}
\newcommand{\VariableTok}[1]{\textcolor[rgb]{0.00,0.00,0.00}{#1}}
\newcommand{\ControlFlowTok}[1]{\textcolor[rgb]{0.13,0.29,0.53}{\textbf{#1}}}
\newcommand{\OperatorTok}[1]{\textcolor[rgb]{0.81,0.36,0.00}{\textbf{#1}}}
\newcommand{\BuiltInTok}[1]{#1}
\newcommand{\ExtensionTok}[1]{#1}
\newcommand{\PreprocessorTok}[1]{\textcolor[rgb]{0.56,0.35,0.01}{\textit{#1}}}
\newcommand{\AttributeTok}[1]{\textcolor[rgb]{0.77,0.63,0.00}{#1}}
\newcommand{\RegionMarkerTok}[1]{#1}
\newcommand{\InformationTok}[1]{\textcolor[rgb]{0.56,0.35,0.01}{\textbf{\textit{#1}}}}
\newcommand{\WarningTok}[1]{\textcolor[rgb]{0.56,0.35,0.01}{\textbf{\textit{#1}}}}
\newcommand{\AlertTok}[1]{\textcolor[rgb]{0.94,0.16,0.16}{#1}}
\newcommand{\ErrorTok}[1]{\textcolor[rgb]{0.64,0.00,0.00}{\textbf{#1}}}
\newcommand{\NormalTok}[1]{#1}
\usepackage{graphicx,grffile}
\makeatletter
\def\maxwidth{\ifdim\Gin@nat@width>\linewidth\linewidth\else\Gin@nat@width\fi}
\def\maxheight{\ifdim\Gin@nat@height>\textheight\textheight\else\Gin@nat@height\fi}
\makeatother
% Scale images if necessary, so that they will not overflow the page
% margins by default, and it is still possible to overwrite the defaults
% using explicit options in \includegraphics[width, height, ...]{}
\setkeys{Gin}{width=\maxwidth,height=\maxheight,keepaspectratio}
\IfFileExists{parskip.sty}{%
\usepackage{parskip}
}{% else
\setlength{\parindent}{0pt}
\setlength{\parskip}{6pt plus 2pt minus 1pt}
}
\setlength{\emergencystretch}{3em}  % prevent overfull lines
\providecommand{\tightlist}{%
  \setlength{\itemsep}{0pt}\setlength{\parskip}{0pt}}
\setcounter{secnumdepth}{0}
% Redefines (sub)paragraphs to behave more like sections
\ifx\paragraph\undefined\else
\let\oldparagraph\paragraph
\renewcommand{\paragraph}[1]{\oldparagraph{#1}\mbox{}}
\fi
\ifx\subparagraph\undefined\else
\let\oldsubparagraph\subparagraph
\renewcommand{\subparagraph}[1]{\oldsubparagraph{#1}\mbox{}}
\fi

%%% Use protect on footnotes to avoid problems with footnotes in titles
\let\rmarkdownfootnote\footnote%
\def\footnote{\protect\rmarkdownfootnote}

%%% Change title format to be more compact
\usepackage{titling}

% Create subtitle command for use in maketitle
\newcommand{\subtitle}[1]{
  \posttitle{
    \begin{center}\large#1\end{center}
    }
}

\setlength{\droptitle}{-2em}

  \title{Clustering Consumers via Topic Modeling: Application to Advertising}
    \pretitle{\vspace{\droptitle}\centering\huge}
  \posttitle{\par}
    \author{}
    \preauthor{}\postauthor{}
    \date{}
    \predate{}\postdate{}
  

\begin{document}
\maketitle

This project poses a question from the perspective of an advertising
agency looking to become more accurate in targeting positive messages to
Facebook users. The agency would like to know: ``How are positive events
and emotions communicated in short form? Are the topics and key points
different across various demographics?'' The agency will use this
information to better serve its advertising clients seeking to attract
customers on Facebook. By collecting data on statuses posted by
different demographics, the agency hopes to classify these users and
communicate relevant ads onto their Facebook page.

This project will leverage the Happy Moments data, which collects
100,000+ responses via Amazon's Mechanical Turk to the question: ``What
made you happy today? Reflect on the past \{24 hours\textbar{}3
months\}, and recall three actual events that happened to you that made
you happy. Write down your happy moment in a complete sentence. Write
three such moments.''

This project will normalize these happy moments for more consumable
analysis, then implement a topic modeling algorithm to categorize these
moments into different topics of conversation. Finally, this project
will cluster the MTurk workers based on their most frequent topic of
conversation in conjunction with relevant demographic attributes to
provide the advertising agency with segment profiles of people that may
post happy moment status on Facebook in the future.

This project and analysis was developed with the following version of R:

\begin{Shaded}
\begin{Highlighting}[]
\KeywordTok{print}\NormalTok{(R.version)}
\end{Highlighting}
\end{Shaded}

\begin{verbatim}
##                _                           
## platform       x86_64-w64-mingw32          
## arch           x86_64                      
## os             mingw32                     
## system         x86_64, mingw32             
## status                                     
## major          3                           
## minor          5.2                         
## year           2018                        
## month          12                          
## day            20                          
## svn rev        75870                       
## language       R                           
## version.string R version 3.5.2 (2018-12-20)
## nickname       Eggshell Igloo
\end{verbatim}

Step \#01: Load the libraries to run the following analysis:

\begin{Shaded}
\begin{Highlighting}[]
\KeywordTok{library}\NormalTok{(dplyr)}
\KeywordTok{library}\NormalTok{(ggplot2)}
\KeywordTok{library}\NormalTok{(tm)}
\KeywordTok{library}\NormalTok{(tidyr)}
\KeywordTok{library}\NormalTok{(proxy)}
\KeywordTok{library}\NormalTok{(topicmodels)}
\KeywordTok{library}\NormalTok{(sqldf)}
\KeywordTok{library}\NormalTok{(matrixStats)}
\KeywordTok{library}\NormalTok{(ldatuning)}
\KeywordTok{library}\NormalTok{(tidytext)}
\KeywordTok{library}\NormalTok{(stringr)}
\KeywordTok{library}\NormalTok{(tibble)}
\KeywordTok{library}\NormalTok{(formattable)}
\KeywordTok{library}\NormalTok{(purrr)}
\end{Highlighting}
\end{Shaded}

Step \#02: Load the data sources from Happy Moments data source:

\begin{Shaded}
\begin{Highlighting}[]
\NormalTok{cleaned_hm <-}\StringTok{ }\KeywordTok{read.csv}\NormalTok{(}\StringTok{'https://raw.githubusercontent.com/rit-public/HappyDB/master/happydb/data/cleaned_hm.csv'}\NormalTok{)}
\NormalTok{demographic <-}\StringTok{ }\KeywordTok{read.csv}\NormalTok{(}\StringTok{'https://raw.githubusercontent.com/rit-public/HappyDB/master/happydb/data/demographic.csv'}\NormalTok{)}
\end{Highlighting}
\end{Shaded}

Step \#03: Create the master file under the following assumptions: 1.
Only include US moments as scope of agency's clients are based in the
United States. 2. Limit the age range to 18-85. Amazon Turk users must
be 18 or older to participate, and elders over 85 years old are not the
target demographic of the agency. 3. Only include users identifying as
male or female to simplify clustering analyses. Transformed binary for
numerical analyses. 4. Only include users with parenthood attribute =
yes or no. Transformed binary for numerical analyses. 5. Transformed
marital status attribute to single or not single for targeted clustering
analysis. Transformed binary for numerical analyses.

\begin{Shaded}
\begin{Highlighting}[]
\CommentTok{#create HMDB central table with appropriate filters and consolidations#}
\NormalTok{hmdb <-}\StringTok{ }\NormalTok{cleaned_hm }\OperatorTok
\StringTok{  }\KeywordTok{inner_join}\NormalTok{(demographic, }\DataTypeTok{by =} \StringTok{'wid'}\NormalTok{) }\OperatorTok
\StringTok{  }\KeywordTok{filter}\NormalTok{(country }\OperatorTok{==}\StringTok{ 'USA'}\NormalTok{) }\OperatorTok
\StringTok{  }\KeywordTok{mutate}\NormalTok{(}\DataTypeTok{age =} \KeywordTok{as.numeric}\NormalTok{(age)) }\OperatorTok
\StringTok{  }\KeywordTok{filter}\NormalTok{(age }\OperatorTok{>=}\StringTok{ }\DecValTok{18} \OperatorTok{&}\StringTok{ }\NormalTok{age }\OperatorTok{<=}\StringTok{ }\DecValTok{85}\NormalTok{) }\OperatorTok
\StringTok{  }\KeywordTok{filter}\NormalTok{(gender }\OperatorTok\StringTok{ }\KeywordTok{c}\NormalTok{(}\StringTok{"m"}\NormalTok{, }\StringTok{"f"}\NormalTok{)) }\OperatorTok
\StringTok{  }\KeywordTok{mutate}\NormalTok{(}\DataTypeTok{gender =}\NormalTok{ (gender }\OperatorTok{==}\StringTok{ 'f'}\NormalTok{)}\OperatorTok{*}\DecValTok{1}\NormalTok{) }\OperatorTok
\StringTok{  }\KeywordTok{filter}\NormalTok{(parenthood }\OperatorTok\StringTok{ }\KeywordTok{c}\NormalTok{(}\StringTok{"n"}\NormalTok{, }\StringTok{"y"}\NormalTok{)) }\OperatorTok
\StringTok{  }\KeywordTok{mutate}\NormalTok{(}\DataTypeTok{parenthood =}\NormalTok{ (parenthood }\OperatorTok{==}\StringTok{ 'y'}\NormalTok{)}\OperatorTok{*}\DecValTok{1}\NormalTok{) }\OperatorTok
\StringTok{  }\KeywordTok{mutate}\NormalTok{(}\DataTypeTok{marital =}\NormalTok{ (marital }\OperatorTok{==}\StringTok{ 'single'}\NormalTok{)}\OperatorTok{*}\DecValTok{1}\NormalTok{) }\OperatorTok
\StringTok{  }\KeywordTok{mutate}\NormalTok{(}\DataTypeTok{id =} \KeywordTok{row_number}\NormalTok{())}
\end{Highlighting}
\end{Shaded}

Step \#04: Pre-process the happy moments so that they are better
prepared for analysis. This includes removing extra spaces, numbers,
stop words (as called by the tm library), and all punctuation.

\begin{Shaded}
\begin{Highlighting}[]
\CommentTok{#create and process a corpus of all moments in scope#}
\NormalTok{corpus <-}\StringTok{ }\KeywordTok{VCorpus}\NormalTok{(}\KeywordTok{VectorSource}\NormalTok{(}\KeywordTok{tolower}\NormalTok{(hmdb}\OperatorTok{$}\NormalTok{cleaned_hm))) }\OperatorTok
\StringTok{  }\KeywordTok{tm_map}\NormalTok{(removePunctuation )}\OperatorTok
\StringTok{  }\KeywordTok{tm_map}\NormalTok{(removeNumbers) }\OperatorTok
\StringTok{  }\KeywordTok{tm_map}\NormalTok{(stripWhitespace) }\OperatorTok
\StringTok{  }\KeywordTok{tm_map}\NormalTok{(removeWords, }\KeywordTok{stopwords}\NormalTok{(}\StringTok{'english'}\NormalTok{))}

\CommentTok{#stem words within each moment to obtain the core word for lookup#}
\NormalTok{stem_hmid <-}\StringTok{ }\KeywordTok{tidy}\NormalTok{(}\KeywordTok{tm_map}\NormalTok{(corpus, stemDocument)) }\OperatorTok
\StringTok{  }\KeywordTok{select}\NormalTok{(text)}

\CommentTok{#create master list of words in the corpus - this serves as the dictionary reference document to compare the stemmed words against#}
\NormalTok{stem_words <-}\StringTok{ }\KeywordTok{tidy}\NormalTok{(corpus) }\OperatorTok
\StringTok{  }\KeywordTok{select}\NormalTok{(text) }\OperatorTok
\StringTok{  }\KeywordTok{unnest_tokens}\NormalTok{(words, text)}

\CommentTok{#compare the stemmed words to the list of words in the corpus - match the most common word associated with each stem and create a unique list of stems and references words#}
\NormalTok{word_reference <-}\StringTok{ }\NormalTok{stem_hmid }\OperatorTok
\StringTok{                    }\KeywordTok{unnest_tokens}\NormalTok{(stems, text) }\OperatorTok
\StringTok{                    }\KeywordTok{bind_cols}\NormalTok{(stem_words) }\OperatorTok
\StringTok{                    }\KeywordTok{anti_join}\NormalTok{(stop_words, }\DataTypeTok{by =} \KeywordTok{c}\NormalTok{(}\StringTok{"words"}\NormalTok{ =}\StringTok{ "word"}\NormalTok{)) }\OperatorTok
\StringTok{                    }\KeywordTok{group_by}\NormalTok{(stems, words) }\OperatorTok
\StringTok{                    }\KeywordTok{summarize}\NormalTok{(}\DataTypeTok{freq =} \KeywordTok{n}\NormalTok{()) }\OperatorTok
\StringTok{                    }\KeywordTok{mutate}\NormalTok{(}\DataTypeTok{clean =}\NormalTok{ words[}\KeywordTok{which.max}\NormalTok{(freq)]) }\OperatorTok
\StringTok{                    }\KeywordTok{distinct}\NormalTok{(stems, clean)}

\CommentTok{#leverage the cleaned unique words to re-create the happy moments with cleaned words - call the cleaned moments: processed_moment#}
\NormalTok{processed <-}\StringTok{ }\NormalTok{stem_hmid }\OperatorTok
\StringTok{            }\KeywordTok{mutate}\NormalTok{(}\DataTypeTok{id =} \KeywordTok{row_number}\NormalTok{()) }\OperatorTok
\StringTok{            }\KeywordTok{unnest_tokens}\NormalTok{(stems, text) }\OperatorTok
\StringTok{            }\KeywordTok{left_join}\NormalTok{(word_reference, }\DataTypeTok{by =} \StringTok{'stems'}\NormalTok{) }\OperatorTok
\StringTok{            }\KeywordTok{filter}\NormalTok{(}\OperatorTok{!}\KeywordTok{is.na}\NormalTok{(clean)) }\OperatorTok
\StringTok{            }\KeywordTok{group_by}\NormalTok{(id) }\OperatorTok
\StringTok{            }\KeywordTok{summarize}\NormalTok{(}\DataTypeTok{processed_moment =} \KeywordTok{str_c}\NormalTok{(clean, }\DataTypeTok{collapse =} \StringTok{" "}\NormalTok{))}

\CommentTok{#join the processed_moment with the rest of the data available about the moment and assoicated worker - eliminate one word moments for easier processing in the document term matrix - select relevant attributes for next step analysis#}
\NormalTok{hmdb <-}\StringTok{ }\NormalTok{hmdb }\OperatorTok\StringTok{ }
\StringTok{          }\KeywordTok{inner_join}\NormalTok{(processed, }\DataTypeTok{by =} \StringTok{'id'}\NormalTok{) }\OperatorTok
\StringTok{          }\KeywordTok{filter}\NormalTok{(num_sentence }\OperatorTok{<=}\StringTok{ }\DecValTok{5}\NormalTok{) }\OperatorTok
\StringTok{          }\KeywordTok{filter}\NormalTok{((}\KeywordTok{sapply}\NormalTok{(}\KeywordTok{strsplit}\NormalTok{(}\KeywordTok{as.character}\NormalTok{(processed_moment), }\StringTok{" "}\NormalTok{), length) }\OperatorTok{>}\StringTok{ }\DecValTok{1}\NormalTok{) }\OperatorTok{==}\StringTok{ }\OtherTok{TRUE}\NormalTok{) }\OperatorTok
\StringTok{          }\KeywordTok{select}\NormalTok{(hmid, wid, id, processed_moment, num_sentence, predicted_category, }
\NormalTok{                 age, marital, parenthood)}
\end{Highlighting}
\end{Shaded}

Step \#05: Create a Document Term Matrix that associates all words with
all documents in a corpus of cleaned happy moments. This is a required
input to the Latent Dirichlet allocation (LDA) method generated in the
next step.

\begin{Shaded}
\begin{Highlighting}[]
\CommentTok{#Create a corpus of processed happy moments - leverage this to create a document term matrix for all processed moments#}
\NormalTok{corpus_processed <-}\StringTok{ }\KeywordTok{Corpus}\NormalTok{(}\KeywordTok{VectorSource}\NormalTok{(hmdb}\OperatorTok{$}\NormalTok{processed_moment)) }\CommentTok{#corpus}
\NormalTok{dtm <-}\StringTok{ }\KeywordTok{DocumentTermMatrix}\NormalTok{(corpus_processed) }\CommentTok{#document term matrix}
\end{Highlighting}
\end{Shaded}

Step \#06: Execute the Latent Dirichlet allocation (LDA) method with the
tuned number of topics as discussed in step 5. The LDA method for topic
modeling with group happy moments into topics of conversations. Below
are parameters for the LDA model, and descriptions of why their values
were selected in the context of the business question and data set:
burnin: 1000 (eliminate the first 1000 iterations of the LDA model.
Because the model randomly assignes topics to words and documents early
on in the iterations, these iterations are not very useful and can be
excluded.) iter: 5000 (after the first 1000 iterations, execute 5000
iterations of the LDA model) thin: 50 (only collect every 50th iteration
of the LDA sequence - because the difference between each individual
iteration is small, we only need to take occasional iterations to
converge the model.) seed: 1000,1001 (seed randomization) nstart: 2
(execute the entire sequence twice) best: TRUE (take the best topics for
final convergence)

\begin{Shaded}
\begin{Highlighting}[]
\CommentTok{#enter parameters for LDA model and execute#}
\KeywordTok{set.seed}\NormalTok{(}\DecValTok{3000}\NormalTok{)}
\NormalTok{lda <-}\StringTok{ }\KeywordTok{LDA}\NormalTok{(dtm, }
           \DataTypeTok{k =} \DecValTok{8}\NormalTok{, }
           \DataTypeTok{method =} \StringTok{'Gibbs'}\NormalTok{,}
           \DataTypeTok{control =} \KeywordTok{list}\NormalTok{(}\DataTypeTok{burnin =} \DecValTok{1000}\NormalTok{,}
                          \DataTypeTok{iter =} \DecValTok{5000}\NormalTok{,}
                          \DataTypeTok{thin =} \DecValTok{50}\NormalTok{,}
                          \DataTypeTok{seed =} \KeywordTok{list}\NormalTok{(}\DecValTok{1000}\NormalTok{,}\DecValTok{1001}\NormalTok{),}
                          \DataTypeTok{nstart =} \DecValTok{2}\NormalTok{,}
                          \DataTypeTok{best =} \OtherTok{TRUE}\NormalTok{))}


\CommentTok{#create aggregate tables out of the LDA output for easier analysis - incorporate results to master HMDB table#}
\NormalTok{topics_words <-}\StringTok{ }\KeywordTok{tidy}\NormalTok{(lda, }\DataTypeTok{matrix =} \StringTok{'beta'}\NormalTok{)}
\NormalTok{topics_docs <-}\StringTok{ }\KeywordTok{tidy}\NormalTok{(lda, }\DataTypeTok{matrix =} \StringTok{'gamma'}\NormalTok{)}
\NormalTok{topics_docs <-}\StringTok{ }\NormalTok{topics_docs }\OperatorTok\StringTok{ }\KeywordTok{spread}\NormalTok{(topic, gamma)}
\NormalTok{topics_docs}\OperatorTok{$}\NormalTok{document <-}\StringTok{ }\KeywordTok{as.integer}\NormalTok{(topics_docs}\OperatorTok{$}\NormalTok{document)}
\NormalTok{hmdb <-}\StringTok{ }\NormalTok{hmdb }\OperatorTok\StringTok{ }\KeywordTok{inner_join}\NormalTok{(topics_docs, }\DataTypeTok{by =} \KeywordTok{c}\NormalTok{(}\StringTok{'id'}\NormalTok{ =}\StringTok{ 'document'}\NormalTok{))}
\end{Highlighting}
\end{Shaded}

Step \#07: Verify that the LDA model has converged, and visualize the
beta for the most common words in each topic. The beta for a given word
is defined as the probability of that term being generated from that
topic.

\begin{Shaded}
\begin{Highlighting}[]
\CommentTok{#identify the top terms in each topic based on their beta percentage#}
\NormalTok{topics_words }\OperatorTok
\StringTok{  }\KeywordTok{group_by}\NormalTok{(topic) }\OperatorTok
\StringTok{  }\KeywordTok{top_n}\NormalTok{(}\DecValTok{8}\NormalTok{, beta) }\OperatorTok
\StringTok{  }\KeywordTok{ungroup}\NormalTok{() }\OperatorTok
\StringTok{  }\KeywordTok{arrange}\NormalTok{(topic, }\OperatorTok{-}\NormalTok{beta) }\OperatorTok
\StringTok{  }\KeywordTok{mutate}\NormalTok{(}\DataTypeTok{term =} \KeywordTok{reorder}\NormalTok{(term, beta)) }\OperatorTok
\StringTok{  }\KeywordTok{ggplot}\NormalTok{(}\KeywordTok{aes}\NormalTok{(term, beta, }\DataTypeTok{fill =} \KeywordTok{factor}\NormalTok{(topic))) }\OperatorTok{+}
\StringTok{  }\KeywordTok{geom_col}\NormalTok{(}\DataTypeTok{show.legend =} \OtherTok{FALSE}\NormalTok{) }\OperatorTok{+}
\StringTok{  }\KeywordTok{facet_wrap}\NormalTok{(}\OperatorTok{~}\StringTok{ }\NormalTok{topic, }\DataTypeTok{scales =} \StringTok{"free"}\NormalTok{) }\OperatorTok{+}
\StringTok{  }\KeywordTok{coord_flip}\NormalTok{()}
\end{Highlighting}
\end{Shaded}

\includegraphics{main_files/figure-latex/unnamed-chunk-8-1.pdf}

Step \#08: With topics created, create an MTurk level of detail data
table that calculates the mean gamma across each topic for each WID. For
simplification, the greatest average gamma value will be denoted the
most common topic for that given WID. Each WID is therefore assigned one
topic. Visualize the counts of WIDs across each of the 8 topics.

\begin{Shaded}
\begin{Highlighting}[]
\CommentTok{#create MTurk level of detail table - select most common topic per person in binary form#}
\NormalTok{wid_lod <-}\StringTok{ }\NormalTok{hmdb }\OperatorTok
\StringTok{            }\KeywordTok{group_by}\NormalTok{(wid) }\OperatorTok
\StringTok{            }\KeywordTok{summarize}\NormalTok{(}\DataTypeTok{age =} \KeywordTok{max}\NormalTok{(age),}
                      \DataTypeTok{parenthood =} \KeywordTok{max}\NormalTok{(parenthood),}
                      \DataTypeTok{marital =} \KeywordTok{max}\NormalTok{(marital),}
                      \DataTypeTok{topic_1 =} \KeywordTok{mean}\NormalTok{(}\StringTok{`}\DataTypeTok{1}\StringTok{`}\NormalTok{),}
                      \DataTypeTok{topic_2 =} \KeywordTok{mean}\NormalTok{(}\StringTok{`}\DataTypeTok{2}\StringTok{`}\NormalTok{),}
                      \DataTypeTok{topic_3 =} \KeywordTok{mean}\NormalTok{(}\StringTok{`}\DataTypeTok{3}\StringTok{`}\NormalTok{),}
                      \DataTypeTok{topic_4 =} \KeywordTok{mean}\NormalTok{(}\StringTok{`}\DataTypeTok{4}\StringTok{`}\NormalTok{),}
                      \DataTypeTok{topic_5 =} \KeywordTok{mean}\NormalTok{(}\StringTok{`}\DataTypeTok{5}\StringTok{`}\NormalTok{),}
                      \DataTypeTok{topic_6 =} \KeywordTok{mean}\NormalTok{(}\StringTok{`}\DataTypeTok{6}\StringTok{`}\NormalTok{),}
                      \DataTypeTok{topic_7 =} \KeywordTok{mean}\NormalTok{(}\StringTok{`}\DataTypeTok{7}\StringTok{`}\NormalTok{),}
                      \DataTypeTok{topic_8 =} \KeywordTok{mean}\NormalTok{(}\StringTok{`}\DataTypeTok{8}\StringTok{`}\NormalTok{))}
\NormalTok{wid_binary <-}\StringTok{ }\KeywordTok{as.data.frame}\NormalTok{(}\KeywordTok{t}\NormalTok{(}\KeywordTok{apply}\NormalTok{(wid_lod[,}\DecValTok{5}\OperatorTok{:}\DecValTok{12}\NormalTok{], }\DecValTok{1}\NormalTok{, }\ControlFlowTok{function}\NormalTok{(x) }\KeywordTok{as.numeric}\NormalTok{(x }\OperatorTok{==}\StringTok{ }\KeywordTok{max}\NormalTok{(x)))))}
\NormalTok{wid_lod <-}\StringTok{ }\NormalTok{wid_lod }\OperatorTok
\StringTok{            }\KeywordTok{cbind}\NormalTok{(wid_binary) }\OperatorTok
\StringTok{            }\KeywordTok{select}\NormalTok{(wid, age, parenthood, marital, V1, V2, V3, V4, V5, V6, V7, V8)}
\KeywordTok{colnames}\NormalTok{(wid_lod) <-}\StringTok{  }\KeywordTok{c}\NormalTok{(}\StringTok{'wid'}\NormalTok{, }\StringTok{'age'}\NormalTok{, }\StringTok{'parent'}\NormalTok{, }\StringTok{'single'}\NormalTok{, }\StringTok{'seeingfriends'}\NormalTok{, }\StringTok{'household'}\NormalTok{,}
                        \StringTok{'accomplishment'}\NormalTok{, }\StringTok{'entertainment'}\NormalTok{, }\StringTok{'family'}\NormalTok{, }\StringTok{'time'}\NormalTok{, }\StringTok{'general'}\NormalTok{, }\StringTok{'food'}\NormalTok{)}

\CommentTok{#remove WID as a clustering feature#}
\KeywordTok{rownames}\NormalTok{(wid_lod) =}\StringTok{ }\NormalTok{wid_lod}\OperatorTok{$}\NormalTok{wid}
\NormalTok{wid_lod}\OperatorTok{$}\NormalTok{wid =}\StringTok{ }\OtherTok{NULL}

\CommentTok{#visualize the count of MTurk wordkers associated with each most common topic#}
\NormalTok{wid_lod[}\DecValTok{4}\OperatorTok{:}\DecValTok{11}\NormalTok{] }\OperatorTok\StringTok{ }\KeywordTok{mutate_all}\NormalTok{(}\KeywordTok{funs}\NormalTok{(sum), }\DataTypeTok{na.rm =} \OtherTok{TRUE}\NormalTok{) }\OperatorTok
\StringTok{  }\KeywordTok{gather}\NormalTok{(}\DataTypeTok{key=}\NormalTok{topic, }\DataTypeTok{value=}\NormalTok{wid_count) }\OperatorTok\StringTok{ }
\StringTok{  }\KeywordTok{ggplot}\NormalTok{(}\KeywordTok{aes}\NormalTok{(}\DataTypeTok{x=}\NormalTok{topic)) }\OperatorTok{+}\StringTok{ }
\StringTok{  }\KeywordTok{geom_bar}\NormalTok{(}\KeywordTok{aes}\NormalTok{(}\DataTypeTok{x =}\NormalTok{ topic, }\DataTypeTok{y =}\NormalTok{ wid_count), }\DataTypeTok{fill =} \StringTok{'navy'}\NormalTok{, }\DataTypeTok{position =} \StringTok{"dodge"}\NormalTok{, }\DataTypeTok{stat =} \StringTok{"identity"}\NormalTok{) }\OperatorTok{+}
\StringTok{  }\KeywordTok{geom_text}\NormalTok{(}\KeywordTok{aes}\NormalTok{(}\DataTypeTok{y =}\NormalTok{ wid_count, }\DataTypeTok{label =}\NormalTok{ wid_count), }\DataTypeTok{nudge_y =} \DecValTok{75}\NormalTok{) }\OperatorTok{+}
\StringTok{  }\KeywordTok{coord_flip}\NormalTok{() }\OperatorTok{+}
\StringTok{  }\KeywordTok{ggtitle}\NormalTok{(}\StringTok{'Count of MTurk Workers by Most Common Topic Classification'}\NormalTok{) }\OperatorTok{+}
\StringTok{  }\KeywordTok{ylab}\NormalTok{(}\StringTok{'Count of MTurk Workers'}\NormalTok{) }\OperatorTok{+}
\StringTok{  }\KeywordTok{xlab}\NormalTok{(}\StringTok{'Topics'}\NormalTok{)}
\end{Highlighting}
\end{Shaded}

\includegraphics{main_files/figure-latex/unnamed-chunk-9-1.pdf} Step
\#09: Test for the appropriate number of clusters of MTurk workers based
on their topic classifications and selected demographic data (age,
marital status, parenthood status). This tests different k (number of
clusters) in a K-Means model and examines the between sum-of-squares
(distance between different clusters) and the within sum-of-squares
(distance within data points in a single cluster). The optimal solution
maximizes the between sum-of-squares and minimizes the within
sum-of-squares, while using the lowest number of clusters for user
consumption. This logic will recommend using six clusters.

\begin{Shaded}
\begin{Highlighting}[]
\CommentTok{#test different values of K#}
\NormalTok{iterations <-}\StringTok{ }\KeywordTok{tibble}\NormalTok{(}\DataTypeTok{k =} \DecValTok{1}\OperatorTok{:}\DecValTok{15}\NormalTok{) }\OperatorTok
\StringTok{  }\KeywordTok{mutate}\NormalTok{(}
    \DataTypeTok{kclust =} \KeywordTok{map}\NormalTok{(k, }\OperatorTok{~}\KeywordTok{kmeans}\NormalTok{(wid_lod, .x)),}
    \DataTypeTok{glanced =} \KeywordTok{map}\NormalTok{(kclust, glance))}
\NormalTok{iterations }\OperatorTok
\StringTok{  }\KeywordTok{unnest}\NormalTok{(glanced, }\DataTypeTok{.drop =} \OtherTok{TRUE}\NormalTok{) }\OperatorTok
\StringTok{  }\KeywordTok{ggplot}\NormalTok{(}\KeywordTok{aes}\NormalTok{(}\DataTypeTok{x =}\NormalTok{ k)) }\OperatorTok{+}
\StringTok{  }\KeywordTok{geom_line}\NormalTok{(}\DataTypeTok{size =} \DecValTok{2}\NormalTok{, }\KeywordTok{aes}\NormalTok{(}\DataTypeTok{y =}\NormalTok{ tot.withinss, }\DataTypeTok{color =} \StringTok{'within SS'}\NormalTok{)) }\OperatorTok{+}
\StringTok{  }\KeywordTok{geom_line}\NormalTok{(}\DataTypeTok{size =} \DecValTok{2}\NormalTok{, }\KeywordTok{aes}\NormalTok{(}\DataTypeTok{y =}\NormalTok{ betweenss, }\DataTypeTok{color =} \StringTok{'between SS'}\NormalTok{)) }\OperatorTok{+}
\StringTok{  }\KeywordTok{geom_vline}\NormalTok{(}\DataTypeTok{size =} \FloatTok{1.5}\NormalTok{, }\KeywordTok{aes}\NormalTok{(}\DataTypeTok{xintercept =} \DecValTok{6}\NormalTok{)) }\OperatorTok{+}
\StringTok{  }\KeywordTok{ggtitle}\NormalTok{(}\StringTok{'Between vs. Within Sum of Squares Cluster Comparison'}\NormalTok{) }\OperatorTok{+}
\StringTok{  }\KeywordTok{ylab}\NormalTok{(}\StringTok{'Sum of Squares'}\NormalTok{) }\OperatorTok{+}
\StringTok{  }\KeywordTok{xlab}\NormalTok{(}\StringTok{'K (# of Clusters)'}\NormalTok{)}
\end{Highlighting}
\end{Shaded}

\includegraphics{main_files/figure-latex/unnamed-chunk-10-1.pdf}

Step \#10: Conduct K-Means cluster analysis on the MTurk workers
leveraging their topics and demographic data points. Gender was not
included as to focus less on gender specific clustering and happy
moments and focus more on life stages and how these topics drive
conversation. Relevant parameter descriptions are provided below for
reference: seed = to ensure reproducibility k = 6 (number of clusters as
determined from pervious step) method = Hartigan-Wong (common K-Means
algorithm)

\begin{Shaded}
\begin{Highlighting}[]
\CommentTok{#execute clusters for given value of K#}
\KeywordTok{set.seed}\NormalTok{(}\DecValTok{5000}\NormalTok{)}
\NormalTok{wid_clusters <-}\StringTok{ }\KeywordTok{kmeans}\NormalTok{(wid_lod, }\DecValTok{6}\NormalTok{, }\DataTypeTok{iter.max =} \DecValTok{10}\NormalTok{, }\DataTypeTok{nstart =} \DecValTok{1}\NormalTok{,}
       \DataTypeTok{algorithm =} \KeywordTok{c}\NormalTok{(}\StringTok{"Hartigan-Wong"}\NormalTok{), }\DataTypeTok{trace=}\OtherTok{FALSE}\NormalTok{)}

\CommentTok{#create summary cluster table and format#}
\NormalTok{cluster_summary <-}\StringTok{ }\KeywordTok{as.data.frame}\NormalTok{(wid_clusters}\OperatorTok{$}\NormalTok{centers)}
\NormalTok{cluster_summary <-}\StringTok{ }\NormalTok{cluster_summary }\OperatorTok
\StringTok{                    }\KeywordTok{mutate}\NormalTok{(}\DataTypeTok{people =} \KeywordTok{c}\NormalTok{(}\DecValTok{1225}\NormalTok{,}\DecValTok{1602}\NormalTok{,}\DecValTok{804}\NormalTok{,}\DecValTok{571}\NormalTok{,}\DecValTok{1879}\NormalTok{,}\DecValTok{1163}\NormalTok{)) }\OperatorTok
\StringTok{                    }\KeywordTok{mutate}\NormalTok{(}\DataTypeTok{names=}\KeywordTok{c}\NormalTok{(}\StringTok{'striving_careers'}\NormalTok{,}
                                   \StringTok{'early_parents'}\NormalTok{,}
                                   \StringTok{'early_retirees'}\NormalTok{,}
                                   \StringTok{'family_heads'}\NormalTok{,}
                                   \StringTok{'fresh_grads'}\NormalTok{,}
                                   \StringTok{'young_professionals'}\NormalTok{))}

\CommentTok{#normalize summary fields for consumable view#}
\NormalTok{cluster_summary <-}\StringTok{ }\NormalTok{cluster_summary[,}\KeywordTok{c}\NormalTok{(}\DecValTok{13}\NormalTok{,}\DecValTok{12}\NormalTok{,}\DecValTok{1}\NormalTok{,}\DecValTok{2}\NormalTok{,}\DecValTok{3}\NormalTok{,}\DecValTok{4}\NormalTok{,}\DecValTok{5}\NormalTok{,}\DecValTok{6}\NormalTok{,}\DecValTok{7}\NormalTok{,}\DecValTok{8}\NormalTok{,}\DecValTok{9}\NormalTok{,}\DecValTok{10}\NormalTok{,}\DecValTok{11}\NormalTok{)]}
\NormalTok{cluster_summary}\OperatorTok{$}\NormalTok{age <-}\StringTok{ }\KeywordTok{round}\NormalTok{(cluster_summary}\OperatorTok{$}\NormalTok{age, }\DataTypeTok{digits =} \DecValTok{1}\NormalTok{)}
\NormalTok{cluster_summary}\OperatorTok{$}\NormalTok{parent <-}\StringTok{ }\KeywordTok{round}\NormalTok{(cluster_summary}\OperatorTok{$}\NormalTok{parent, }\DataTypeTok{digits =} \DecValTok{2}\NormalTok{)}
\NormalTok{cluster_summary}\OperatorTok{$}\NormalTok{single <-}\StringTok{ }\KeywordTok{round}\NormalTok{(cluster_summary}\OperatorTok{$}\NormalTok{single, }\DataTypeTok{digits =} \DecValTok{2}\NormalTok{)}
\NormalTok{cluster_summary}\OperatorTok{$}\NormalTok{seeingfriends <-}\StringTok{ }\KeywordTok{round}\NormalTok{(cluster_summary}\OperatorTok{$}\NormalTok{seeingfriends, }\DataTypeTok{digits =} \DecValTok{4}\NormalTok{)}
\NormalTok{cluster_summary}\OperatorTok{$}\NormalTok{household <-}\StringTok{ }\KeywordTok{round}\NormalTok{(cluster_summary}\OperatorTok{$}\NormalTok{household, }\DataTypeTok{digits =} \DecValTok{4}\NormalTok{)}
\NormalTok{cluster_summary}\OperatorTok{$}\NormalTok{accomplishment <-}\StringTok{ }\KeywordTok{round}\NormalTok{(cluster_summary}\OperatorTok{$}\NormalTok{accomplishment, }\DataTypeTok{digits =} \DecValTok{4}\NormalTok{)}
\NormalTok{cluster_summary}\OperatorTok{$}\NormalTok{entertainment <-}\StringTok{ }\KeywordTok{round}\NormalTok{(cluster_summary}\OperatorTok{$}\NormalTok{entertainment, }\DataTypeTok{digits =} \DecValTok{4}\NormalTok{)}
\NormalTok{cluster_summary}\OperatorTok{$}\NormalTok{family <-}\StringTok{ }\KeywordTok{round}\NormalTok{(cluster_summary}\OperatorTok{$}\NormalTok{family, }\DataTypeTok{digits =} \DecValTok{4}\NormalTok{)}
\NormalTok{cluster_summary}\OperatorTok{$}\NormalTok{time <-}\StringTok{ }\KeywordTok{round}\NormalTok{(cluster_summary}\OperatorTok{$}\NormalTok{time, }\DataTypeTok{digits =} \DecValTok{4}\NormalTok{)}
\NormalTok{cluster_summary}\OperatorTok{$}\NormalTok{general <-}\StringTok{ }\KeywordTok{round}\NormalTok{(cluster_summary}\OperatorTok{$}\NormalTok{general, }\DataTypeTok{digits =} \DecValTok{4}\NormalTok{)}
\NormalTok{cluster_summary}\OperatorTok{$}\NormalTok{food <-}\StringTok{ }\KeywordTok{round}\NormalTok{(cluster_summary}\OperatorTok{$}\NormalTok{food, }\DataTypeTok{digits =} \DecValTok{4}\NormalTok{)}
\end{Highlighting}
\end{Shaded}

Step \#11: Format and display the cluster output for advertising agency
analysis. This table provides details about each MTurk Worker segment,
including the \% of workers in that segment by most common topic. While
the percentages across each topic and cluster do not vary drastically,
there are notable differences across segments and topics that the agency
can use to inform their ad presence online. At scale on a platform like
Facebook, knowing that a particular segment gains happiness from a
partiular topic more than others is extremely valuable for positive
ethos flavored advertising.

\begin{Shaded}
\begin{Highlighting}[]
\KeywordTok{formattable}\NormalTok{(cluster_summary, }
            \DataTypeTok{align =} \KeywordTok{c}\NormalTok{(}\StringTok{'l'}\NormalTok{,}\StringTok{'c'}\NormalTok{,}\StringTok{'c'}\NormalTok{,}\StringTok{'c'}\NormalTok{,}\StringTok{'c'}\NormalTok{,}\StringTok{'c'}\NormalTok{,}\StringTok{'c'}\NormalTok{,}\StringTok{'c'}\NormalTok{,}\StringTok{'c'}\NormalTok{,}\StringTok{'c'}\NormalTok{,}\StringTok{'c'}\NormalTok{,}\StringTok{'c'}\NormalTok{,}\StringTok{'c'}\NormalTok{),}
            \KeywordTok{list}\NormalTok{(}\StringTok{`}\DataTypeTok{names}\StringTok{`}\NormalTok{ =}\StringTok{ }\KeywordTok{formatter}\NormalTok{(}\StringTok{"span"}\NormalTok{, }\DataTypeTok{style =} \OperatorTok{~}\StringTok{ }\KeywordTok{style}\NormalTok{(}\DataTypeTok{color =} \StringTok{"grey"}\NormalTok{,}\DataTypeTok{font.weight =} \StringTok{"bold"}\NormalTok{)),}
                  \StringTok{`}\DataTypeTok{seeingfriends}\StringTok{`}\NormalTok{ =}\StringTok{ }\KeywordTok{color_tile}\NormalTok{(}\StringTok{'#ff7f7f'}\NormalTok{, }\StringTok{'#71CA97'}\NormalTok{),}
                  \StringTok{`}\DataTypeTok{household}\StringTok{`}\NormalTok{ =}\StringTok{ }\KeywordTok{color_tile}\NormalTok{(}\StringTok{'#ff7f7f'}\NormalTok{, }\StringTok{'#71CA97'}\NormalTok{),}
                  \StringTok{`}\DataTypeTok{accomplishment}\StringTok{`}\NormalTok{ =}\StringTok{ }\KeywordTok{color_tile}\NormalTok{(}\StringTok{'#ff7f7f'}\NormalTok{, }\StringTok{'#71CA97'}\NormalTok{),}
                  \StringTok{`}\DataTypeTok{entertainment}\StringTok{`}\NormalTok{ =}\StringTok{ }\KeywordTok{color_tile}\NormalTok{(}\StringTok{'#ff7f7f'}\NormalTok{, }\StringTok{'#71CA97'}\NormalTok{),}
                  \StringTok{`}\DataTypeTok{family}\StringTok{`}\NormalTok{ =}\StringTok{ }\KeywordTok{color_tile}\NormalTok{(}\StringTok{'#ff7f7f'}\NormalTok{, }\StringTok{'#71CA97'}\NormalTok{),}
                  \StringTok{`}\DataTypeTok{time}\StringTok{`}\NormalTok{ =}\StringTok{ }\KeywordTok{color_tile}\NormalTok{(}\StringTok{'#ff7f7f'}\NormalTok{, }\StringTok{'#71CA97'}\NormalTok{),}
                  \StringTok{`}\DataTypeTok{general}\StringTok{`}\NormalTok{ =}\StringTok{ }\KeywordTok{color_tile}\NormalTok{(}\StringTok{'#ff7f7f'}\NormalTok{, }\StringTok{'#71CA97'}\NormalTok{),}
                  \StringTok{`}\DataTypeTok{food}\StringTok{`}\NormalTok{ =}\StringTok{ }\KeywordTok{color_tile}\NormalTok{(}\StringTok{'#ff7f7f'}\NormalTok{, }\StringTok{'#71CA97'}\NormalTok{)))}
\end{Highlighting}
\end{Shaded}

names

people

age

parent

single

seeingfriends

household

accomplishment

entertainment

family

time

general

food

{striving\_careers }

1225

47.3

0.60

0.33

{0.1273}

{0.1216}

{0.1567}

{0.1437}

{0.1233}

{0.1143}

{0.0890}

{0.1486}

{early\_parents }

1602

37.8

0.48

0.42

{0.1261}

{0.1292}

{0.1542}

{0.1417}

{0.1105}

{0.1242}

{0.0874}

{0.1517}

{early\_retirees }

804

61.2

0.69

0.27

{0.1331}

{0.1381}

{0.1318}

{0.1493}

{0.1480}

{0.1082}

{0.0759}

{0.1393}

{family\_heads }

571

77.3

0.74

0.19

{0.1156}

{0.1366}

{0.1541}

{0.1349}

{0.1436}

{0.0981}

{0.0911}

{0.1331}

{fresh\_grads }

1879

22.6

0.17

0.75

{0.1304}

{0.1086}

{0.1613}

{0.1575}

{0.1261}

{0.1075}

{0.0761}

{0.1543}

{young\_professionals}

1163

29.5

0.33

0.59

{0.1273}

{0.1376}

{0.1625}

{0.1410}

{0.1204}

{0.1118}

{0.0894}

{0.1290}

Brief anecdotes about the clusters (sorted by average age) ``Fresh
Grads'' - an average age of 22, these individuals find happiness in
accomplishment, friends, and food topics. They are not very concerned
with events happening in their household, because they likely are not
the head of household (only 17\% are parents).

``Young Professionals'' - these individuals find happiness in topics
surrounding their achievements and their household activities more than
other clusters. These are younger individuals who are likely progressing
quickly in both their professional and personal lives.

``Early Parents'' - these are younger parents who find happiness in
spending time with their loved ones or on their own.

``Striving Careers'' - these are middle-aged individuals gaining
happiness from achievements in their life, career or otherwise.

``Early Retirees'' - these individuals average 60 years old, and find
happiness in spending time with friends more than other segments.
Additionally, they are interested in their families and household
activities, likely finding more time to re-connect.

``Family Heads'' - the oldest cluster speaks more about family and
events in the household than other segments. These individuals are
likely retired grandparents who gain happiness from their family tree
and spending quality time together.

Step \#12: Export WID details for advertising agency and further
analysis. Please update path to relevant project path for export.

\begin{Shaded}
\begin{Highlighting}[]
\CommentTok{#write MTurk extract and cluster IDs out#}
\KeywordTok{write.csv}\NormalTok{(wid_lod, }\StringTok{'C:/Users/mkars/Documents/GitHub/Spring2019-Proj1-mkarsok/output/wid_lod.csv'}\NormalTok{)}
\CommentTok{#write master moments file out#}
\KeywordTok{write.csv}\NormalTok{(hmdb, }\StringTok{'C:/Users/mkars/Documents/GitHub/Spring2019-Proj1-mkarsok/output/hmdb.csv'}\NormalTok{)}
\end{Highlighting}
\end{Shaded}

citations: topic modeling:
\url{https://cfss.uchicago.edu/fall2016/text02.html} Happy Moments info:
\url{https://rit-public.github.io/HappyDB/} Data: Akari Asai, Sara
Evensen, Behzad Golshan, Alon Halevy, Vivian Li, Andrei Lopatenko,
Daniela Stepanov, Yoshihiko Suhara, Wang-Chiew Tan, Yinzhan Xu, HappyDB:
A Corpus of 100,000 Crowdsourced Happy Moments'', LREC '18, May 2018.


\end{document}
